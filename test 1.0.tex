\documentclass[draft]{article}
\usepackage{shapepar}
% \usepackage{microtype}
\usepackage{multicol}

\title{Ecopoetics Groundwork}
\author{Ira Livingston}
\date{v1.1 \ Mar 2022}

\begin{document}
\maketitle
\subsection{Collaborative Meaning Making}
%custom paragraph shape defined here

I begin with a bit of a poem, the way a rabbi or preacher would start with a biblical verse. Today's sermon is drawn from Wallace Stevens' massively canonical poem "The Idea of Order at Key West," in which two men philosophize as they watch and listen to a woman singing at the seaside, and they are struck by the sense that \par
. . . there was no world for her 
Except the one she sang and, singing, made.
The men are trying to come to terms with what, if anything, the singing (standing for art, language and consciousness generally) does to the world-- its relationship with the sea, and the question of who dances to whose tune. As in Stevens' poem "Anecdote of the Jar," in which the simple placement of an empty glass jar on a hill has somehow organized the wilderness around it and has taken "dominion everywhere," the effect of her singing is both vanishingly subtle and total; the sea and the night sky are harmonized, enchanted, and thrown into mystical perspective by it. \par What can be the effect of human meaning-making on the world, and how do language and art participate in shaping it? An open, high-stakes question for us in the 21st century, and for ecopoetics. \par Part of the point of Stevens' poem seems to be how philosophy-- which Stevens codes as masculine-- falls short of art-- which he codes as feminine-- but the poet, by folding philosophy back into art in the form of the poem, manages to perform a transcending synthesis. It's an old Wordsworthian move: a kind of dialectical masculinism, starting with the binary distinctions of culture and nature, singer and sea. The poet is the woman singing and the men philosophizing. \par Or does he attribute this synthesis to the singer? It's hard to tell. \par The point of "no world for her / Except the one she sang" seems to be that consciousness and language alienate us from the natural world (we are "an unhappy people in a happy world" as Stevens put it elsewhere) and we are thus obliged to make our own world. If singing were a mere elaboration of the rhythms of human breathing, it would remain a sibling phenomenon with the waves, a "heaving speech of air, a summer sound / Repeated in a summer without end," where summer (typically for Stevens) represents the happy and unalienated natural world. So the poem seems to be advancing the (familiar) proposition that human meaning is made in an otherwise meaningless world-- an existentialist idea of how we are challenged to embrace our radical freedom in world-making. Again, this seems to be a replay of the Wordsworthian account of "how the mind of man becomes / A thousand times more beautiful than the earth / On which he dwells" and is "exalted" further to realize itself as being "of quality and fabric more divine." Wordsworth encodes the proposition in quantitative and comparative terms ("more beautiful" and "more divine"): language and consciousness are emergent phenomena, arising from but transcending the natural world. This proposition is part of what Romanticism built into the concept of nature, making it an almost insurmountable conceptual obstacle to the understanding we are trying to access here. This is also why "natural supernaturalism" and "religious naturalism" (forms of secular religion) are part of the problem; why they are so accomodating to the modernity they may seem to oppose. \par In Stevens' poem, the ordering performed by art is cast as a deepening of the world, as when the tilting (physical) masts of sailboats at harbor seem to extend into (imaginary) perspective lines that deepen the night, as if they were positing castles in the air, spiritualizing the world with "ghostlier demarcations" as with the singer's "keener sounds." As in Wordsworth, the punchline seems to be that the human and even the divine are not different in kind from the natural but in degree (keener, ghostlier). This proposition was radical in Wordsworth's time for displacing the hierarchical and stable-for-all-eternity order-of-things embodied in a Great Chain of Being based on differences in kind (as between aristocrats and commoners). Even so, the new order recuperates hierarchy in a dynamic capitalist system in which the middle class constantly earns its dominance as a difference in degree. You get the formula over and over in Wordsworth, as in his famous definition of a poet as "a man speaking to men" but-- wait for it-- also a man "endued with more lively sensibility, more enthusiasm and tenderness, who has a greater knowledge of human nature, and a more comprehensive soul." Again, note the assertion of no difference in kind (though again coded as masculine) followed by repeated insistence on difference in degree that comes as a kind of backlash ("some are more equal than others," as Orwell put it). \par This leaves for us the still-unfinished project of (1) rejection of the recuperated hierarchy (via what can be called anarchism) and (2) reopening the possibility of thinking and enacting equality with radical difference. For this we need to set Wordsworth aside and turn to someone like William Blake, whose life's work can be understood in terms of these projects. \par In its masculinism and modernism and humanist triumphalism, Stevens' poem telegraphs its limitations, but in so doing, it points beyond: "meaning indicates the direction in which it fails." \par The beyond appears in what might be Stevens' last poem, "Of Mere Being" (to which I return briefly in Chapter Three), where human meaning and reason are entirely displaced as the poem beatifically confronts an otherness at the heart of meaning, this time in the form of an other- worldly bird of paradise and its song. This is a version of an otherness I encounter again and again in this book: in the constellation of animals, humans, monsters and divinities that make up Rome's Trevi Fountain (Chapter Three), the sea monster Leviathan and a host of other creatures real and fantastic in the Book of Job (Chapter Five) and the menagerie of oracular animals in Blake's "Auguries of Innocence" (Chapter Six). \par Without denying the psychological resonance of an existential loneliness in which there is no world for us but what we make, this book starts from another premise altogether: that the world is full of meaning, and that meaning and complexity are primordial. \par Meaning, as I understand it, is how things-- often incommensurable things at several removes-- matter to each other. This is how systems evolve; it is an almost tautological account of ecology, and it makes the relationship of language to the world kin to the relationships of other creatures and systems. I will not belabor this point since I have written about it extensively elsewhere, but if you like official-sounding names, you can file it as a form of anti-reductionism: simple elements don't come first and get distributed into complex patterns; pattern and the elements differentiated by patterning co-evolve. The reflective capacity of the mind draws on the same recursivity that builds the world, making the gemlike drop of dew that reflects the world (a favorite image of consciousness and language in metaphysical poetry) more complex than the world it reflects, and making the languaged world more plural for its keener and ghostlier inhabitations. \par Accordingly, while rejecting the humanist/existentialist assertion that "humans make meaning in a meaningless world," I think it is important also to reject the Nietzschean argument that humans come along belatedly, a dispensible afterthought in a universe or an ecology that is already full without us. How could you have respect for the sanctity of any life if you don't include your own (and vice versa)? The slightly longer way of putting this is that, just as a universe in which forces and particles and larger structures have evolved as they have in our universe is fundamentally different than one in which they have evolved differently, a universe with consciousness and language in it is also fundamentally different than one without. This position differs from the humanist exceptionalism of Wordsworth and Stevens: we are co-players and co- makers. So to Stevens' narrator's assertion that "She was the single artificer of the world / In which she sang," I'm tempted just to scrawl NO in the margin, or to suggest a rewrite: "they were a multiple co-maker of the world in which they sang." \par 
\shapepar\diamondshape ASIDE: The Single Artificer. Why does Stevens make such an obviously wrong assertion as "she was the single artificer"? Because one needs it as a mythic manifesto in order to claim a kind of godlike sovereignty for the artist? Why does existentialism insist so obviously wrongly on radical freedom? It's crazy, right? Or do we need this fantasy in order to wrest even a modicum of our own agency from an otherwise all-powerful world, like a surfer on a tsunami? There is a more sophisticated way of justifying the assertion with systems theory's account of the simultaneous openness and closure of complex systems. Again, because I've written extensively about this elsewhere, here's a super-condensed example. Language, in which we build a model of the world, is built entirely of its own differences: the sounds by which one word is distinguished from another in a given language (like the way one word is defined against another) don't signify much if anything in the world at large, even though they are physical sounds like any other physical sounds in the world. We can say much the same about the intricate double-helical arrangements of the four chemicals of DNA, but as you may have noticed, it is both deeply right and deeply wrong to say that "we" are the single artificer of these (as right and wrong as saying they are the single artificer of us). To go further with this you need to start dismantling the dualistic opposition of openness and closure as they are definitive for systems that make their own components.


\end{document}