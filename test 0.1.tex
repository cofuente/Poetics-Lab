% shapepar.ltx   Manual for shapepar.sty
%
% Version 2.0  (Dec 2002)
%
% Copyright (c) 1993,2002    Donald Arseneau
% These definitions may be freely transmitted, reproduced, or modified
% provided that any modifications are clearly identified and this notice
% is left intact.
%


\documentclass[draft]{article}
\usepackage{shapepar}



\renewcommand{\topfraction}{.85} 
\renewcommand{\bottomfraction}{.7}
\renewcommand{\textfraction}{.15} 
\renewcommand{\floatpagefraction}{.66}
\renewcommand{\dbltopfraction}{.66} 
\renewcommand{\dblfloatpagefraction}{.66}
\raggedbottom
\addtolength{\topskip}{0pt plus 8pt}

\newcommand\configpar[6]{\par\penalty-50
 \vskip 0pt plus 12pt \penalty 50 \vskip 0pt plus -12pt
 \begingroup\samepage
 \paragraph*{\textbackslash #1} #2\unskip\strut\\*
 \begin{tabular}{@{}p{.15\linewidth}@{}p{.85\linewidth}@{}}
 Type: \ & \rightskip\fill #3\\
 Default: \ & \rightskip\fill #4\\
 Set with: \ & \rightskip\fill #5\\
 Example: \ & \rightskip\fill #6
 \end{tabular}
 \par \endgroup
}

% Code from: underscore.sty     12-Oct-2001   Donald Arseneau   asnd@triumf.ca
% Make the "_" character print as "\textunderscore" in text.

\makeatletter
\begingroup
 \catcode`\_=\active
 \gdef_{% \relax % No relax gives a small vulnerability in alignments
   \ifx\if@safe@actives\iftrue % must be outermost test!
      \string_%
   \else
      \ifx\protect\@typeset@protect
         \ifmmode \sb \else \BreakableUnderscore \fi
      \else
         \ifx\protect\@unexpandable@protect \noexpand_%
         \else \protect_%
      \fi\fi
    \fi}
\endgroup

% At begin: set catcode; fix \long \ttdefault so I can use it in comparisons; 
\AtBeginDocument{%
  {\immediate\write\@auxout{\catcode\number\string`\_ \string\active}}%
  \catcode\string`\_\string=\active
  \edef\ttdefault{\ttdefault}%
}

\newcommand{\BreakableUnderscore}{\leavevmode\nobreak\hskip\z@skip
 \ifx\f@family\ttdefault \string_\else \textunderscore\fi
 \usc@dischyph\nobreak\hskip\z@skip}

\DeclareRobustCommand{\_}{%
  \ifmmode \nfss@text{\textunderscore}\else \BreakableUnderscore \fi}

\let\usc@dischyph\@dischyph
\makeatother
%  end of underscore.sty extract  %%%%%%%%%%%%%%%%%%%%%%%%%%%%%


\title{shapepar.sty}
\author{Donald Arseneau\\ Vancouver, Canada\\ asnd@triumf.ca}
\date{v 2.0 \ Dec 2002}

\begin{document}
\begin{minipage}{0.2\textwidth}
\begin{tabular}{|c|c|c|}
\hline
 A & B & C \\
\hline
 1 & 2 & 3  \\
\hline 
 4 & 5 & 6 \\
\hline
\end{tabular}
\end{minipage}
\begin{minipage}{0.2\textwidth}
\begin{tabular}{c|c|c}
 A & B & C \\
\hline
 1 & 2 & 3  \\
\hline 
 4 & 5 & 6 \\
\end{tabular}
\end{minipage}
%%%%%%%%%%%%%%%  Active character definitions:

\def\Meaningless#1>{}
\catcode`\"=\active
\def\startV{\leavevmode\begingroup
  \ifdim 0pt=\lastskip\penalty200 \fi
  \catcode`\{11 \catcode`\}11 \catcode`\%11
  \moreV}
\long\def\moreV#1"{%
  \def\LtxCode{#1}%
  \texttt{\ignorespaces
      \expandafter\Meaningless\meaning\LtxCode
      \unskip}%
  \endgroup}
\let"\startV

\catcode`\<=\active
\def<#1>{\ensuremath{\langle\mbox{\textsl{#1}}\rangle}}

%%%%%%%%%%%%%%%  End active character definitions

\maketitle

%\begin{abstract}
\noindent
"\shapepar": a macro to typeset paragraphs in a specific shape. The
size is adjusted automatically so that the entire shape is filled
with text.  This package is for Plain \TeX, \LaTeX, or similar.
%\end{abstract}

\section{\textbackslash shapepar and \textbackslash Shapepar}

The "\shapepar" macro (or `command') is used to typeset paragraphs
of a specified shape, where
the total size is adjusted automatically so that the entire shape is
filled with text, and the shape may include separate pieces and 
holes.  This is distinct from the normal "\parshape" command
which specifies a simple shape \emph{and} a size that may be partially
filled, or over-filled, from top to bottom. In a "\shapepar" there
can be no displayed math, and no `"\vadjust"' material, (including
"\vspace"). This style is mainly
intended for cards, invitations etc., not for whole books!  Although
short paragraphs process much faster, only long paragraphs accurately
fill complex shapes.

These macros work for both \LaTeX\ and plain \TeX.  For \LaTeX, specify
"\usepackage""{shapepar}", or for Plain, "\input" "shapepar.sty".

The command "\shapepar" should be used at the beginning of a paragraph,
and it applies to the entire paragraph.  There is one optional length
parameter: a fixed scale, <scale_len>; and one required parameter:
a description of the shape, <shape_spec>.

\begin{flushleft}
"\shapepar" "["<scale_len>"]""{"<shape_spec>"}" Text of the paragraph
\end{flushleft}

The text of the paragraph is delimited by a blank line or "\par", but
is not literally a \emph{parameter}, so verbatim macros will work there.
If you want to typeset two paragraphs in one shape, then use
"\endgraf" or "\\" to split them.

Ordinarily, the scale is calculated automatically so the pargaraph
best fills the shape.  If a scale length is given, then the shape is
reproduced so one unit of the <shape_spec> equals the <scale_len>, and
the shape is filled with white space after the paragraph text.
The <scale_len> is much like "\unitlength" for the picture environment.
For an application of a fixed scale, see the "\CDlabel" macro.

With the "\shapepar" command, the text will be typeset centered on the
page using the specified shape (specifically, the shape's <h_center>
will be centered on the page; see below). A "\shapepar" should not
break across pages (due to inter-line penalties) but that feature is
not guaranteed.

The "\Shapepar" macro (capital~S) typesets the shaped paragraph in a box
("\vtop" or "\parbox""[t]") without extra horizontal padding. If it occurs
in vertical mode, special care is taken with the line-spacing around
this box, but the line-spacing might be better with plain "\shapepar".
"\Shapepar" is particularly useful for "\fbox" or "\put".

\section{\textbackslash cutout}

A shaped paragraph can be incorporated with the page in a third way:
nestled in a cut-out at the side of the running text.

\begin{flushleft}
"\cutout" "{"<side>"}" "("<h_offset>","<v_offset>")" <settings> "\shapepar" \dots
\end{flushleft}
The <side> argument is required, and must be `l' or `r', indicating
which margin (left or right) the shape should occupy.  By default (in
the absence of other parameters) the shaped paragraph will be placed so
its center-line (<h_center>) is at the specified edge of the running
text, and its first line is level with the first line of the ensuing
text. This position may be changed by specifying the optional horizontal
and vertical offset distances "("<h_offset>","<v_offset>")" in
parentheses.  A positive <h_offset> moves the shape further right, and a
positive <v_offset> moves it down. A negative <v_offset> will move the
shape upwards, where it might overlap preceding text, which will
\emph{not} be cut away to accommodate.

Yes, the text coming after the "\cutout" "\shapepar" is cut out (surprise!)
to fit the shape, leaving a gap of `"\cutoutsep"'.  This length (dimen)
parameter is initialized to 12\,pt; you may set it as you please.
The cutout separation is applied by looking at the paragraph shape
expanded by "\cutoutsep" in all directions, to give the same gap at
every slope. The cutout only lasts for one paragraph.\footnote
{Re-defining \texttt{\string\p ar} for the extent of the cutout is an
attractive feature, but has conflicts with various \TeX\ formats.
It seems best to leave the paragraph control in the hands of the user.
This will likely change in the future.}
To extend the effect further, divide paragraphs with
`"\\[\parskip""]" "\indent"' (\LaTeX) or `"\hfill" "\break" "\indent"'
so \TeX\ does not treat them as separate paragraphs.
Furthermore, the cut-out paragraph should not end in a local
group, so make sure there is an explicit "\par" or blank line
there, outside of braces.

\font\cmsybig=cmsy10 at 1.2in
\newcommand\BigO{{\cmsybig \char10}}
\cutout{r}(-1cm,2\baselineskip)
\setlength\cutoutsep{6pt}
\shapepar[1.7cm]{\circleshape}%
\setlength\unitlength{1cm}%
\begin{picture}(0,0)%
\put(1.0,-1.3){\makebox[0pt]{\BigO}}
\end{picture}
\par



\indent The combination of "\cutout" with fixed scale allows an
entirely different application: producing cutouts for graphics.  First,
produce a rough <shape_spec> for the image to delineate its left and
right borders (you can ignore all internal detail) or use an appropriate
pre-defined shape.  Then use "\includegraphics" as the text of the
shaped paragraph.  There is a difficulty though: we want the image
to take the place of the entire paragraph, not to stand up on the
top line.  There are a number of methods for lowering the graphic
appropriately (and some methods that will not work with "\shapepar");
three that work are:

\begin{flushleft}
"\hfill\makebox[0pt][c]{""\raisebox""{1ex-\height}" \% requires calc.sty  \\*
\qquad" {\includegraphics{"<file>"}}}" \\*
or \\
"\hfill"\\*
"\begin""{minipage}""[t]{0pt}"\\*
"  \vspace {-1ex} % \vspace works here only"\\*
"  \centerline{ \includegraphics{"<file>"} }"\\*
"\end{minipage}%"\\*
"\hfill"\\*
and, most general,\\
"\begin""{picture}""(0,0)"\\*
"\put("$x,y$"){\includegraphics{"<file>"}}"\\*
"\end{picture}"
\end{flushleft}
The first two adjust the height automatically, but assume the
top line of the shape is horizontally centered.  The picture
environment handles positioning more flexibly, but requires
you to provide parameters $x,y$. Just try approximate values,
make a test run, measure the offset, and correct. This should
only require one trial, not an endless cycle. Because the 
reference point for the shaped paragraph is based on the \emph{text} 
it contains, and not just the specified shape, some adjustment
will usually be needed to position a shaped paragraph precisely.

% For example,
%this paragraph begins with
%\begin{verbatim}
%\cutout{r}(-1cm,\baselineskip)
%\setlength\cutoutsep{6pt}
%\shapepar[1cm]{\circleshape}%
%\setlength\unitlength{1cm}%
%\begin{picture}(0,0)
%\put(-2,-2){\BigO}
%\end{picture}
%\end{verbatim}
%(I used a big symbol rather than including a graphics file
%with this manual.)


\section{Shapes}

There are some shapes predefined in shapepar.sty (square, circle,
circle-with-hole, diamond, heart, star and hex-nut) which are used as
examples in the instructions below. Each of these shapes is stored in
a macro, and there is a command to use that shape:
\begin{flushleft}
\begin{tabular}{lll}
"\squareshape"& "\squarepar"& Square \\
"\circleshape"& "\circlepar"& Circle \\
"\CDshape"    & "\CDlabel"  & Circle with circular hole ("\CDlabel" uses\\
              &             & a fixed scale to fit a compact disc)\\
"\diamondshape" & "\diamondpar" & Rhomboid `diamond' ($\diamondsuit$)\\
"\heartshape" & "\heartpar" & Heart (symbolic shape $\heartsuit$)\\
"\starshape"  & "\starpar"  & Five-point star\\
"\nutshape"   & "\nutpar"   & Nut for bolt (hexagon with circular hole)
\end{tabular}
\end{flushleft}
For example, "\heartpar{"<text>"}" performs
"\shapepar" "{\heartshape}" <text>"\ \ $""\heartsuit""$""\par"
(ending the text with a heart symbol).

More shape definitions are provided in separate files named
$*$shape.def.  Look on your disk to be sure, but the
list should include:
\begin{flushleft}
\begin{tabular}{ll}
candleshape  & A burning candle\\
TeXshape     & The \TeX\ logo\\
Canflagshape & The Canadian flag
%\\
% USAshape     & The American flag
\end{tabular}
\end{flushleft}
Please contribute your shapes!

Although defining shapes by hand can be difficult, there are
programs to aid you.

\subsection{proshap.py by Man\-uel Gu\-ti\-er\-rez Al\-ga\-ba}
\gdef\bassshape{{9.4}{0}b{9.4}
\\{0}t{7.6}{6.8}
\\{0.8}t{5.2}{11.6}
\\{1.6}t{4.4}{13.2}
\\{2.4}t{4.0}{14.0}
\\{3.2}t{3.6}{4.933333}st{8.533333}{4.933333}st{13.46666}{4.933333}
\\{4.0}t{3.6}{2.4}t{9.6}{2.8}t{16.4}{2.4}
\\{6.4}t{3.6}{2.0}t{10.0}{2.0}t{16.8}{2.0}
\\{8.8}t{3.6}{2.4}t{9.6}{1.4}st{11.0}{1.4}t{16.4}{2.0}
\\{11.2}t{3.6}{3.4}jt{7.0}{3.4}t{11.6}{3.4}jt{15.0}{3.4}
\\{12.8}t{4.0}{6.0}t{12.0}{6.0}
\\{14.4}t{4.4}{5.2}t{12.4}{5.2}
\\{16.0}t{4.8}{6.0}jt{10.8}{6.0}
\\{16.8}t{5.2}{5.6}st{10.8}{5.6}
\\{17.6}t{5.6}{0.8}t{14.8}{1.2}
\\{19.2}t{6.0}{2.4}t{12.0}{3.6}
\\{20.8}t{6.4}{4.4}jt{10.8}{4.4}
\\{22.4}e{9.4}}


\shapepar\bassshape
proshap.py (ver 1.1) is a python script written by Man\-uel
Gu\-ti\-er\-rez Al\-ga\-ba to produce shape definitions from rough
`ascii art'.  There is no instruction manual, so here are Donald
Arseneau's observations. There is not much of a user interface; look
in proshap.py (which is a plain text file) and see how the various
`test' shapes are defined (note the triple-double quotes).  Choose
one of them, or add a new one, then change the line `test = test3' to select
the desired picture.  Execute `python proshap.py' which will output a
definition of "\bassshape" to the screen and to the file `result.tex'.
The goul\-ish face you see here is the test3 shape.  You should be
aware that the characters in the ascii input are treated as square,
even though they are taller than they are wide, so the output shape
specification will be taller and thinner than the input text.  There
also seems to be a problem with all `bottoms': flat bottoms of text
blocks and of holes are expanded downwards to end at a point.  Compare
this face to the original face in proshap.py.  Warning: These
instructions and observations are probably wrong; the author does not
program in python so can't even read the code properly.  For now, look
for proshap.py bundled with shapepar.sty.

\subsection{proshap.py by Man\-uel Gu\-ti\-er\-rez Al\-ga\-ba}
\begin{minipage}{0.2\textwidth}
\gdef\bassshape{{9.4}{0}b{9.4}
\\{0}t{5.2}{6.8}
\\{4.0}t{5.2}{6.8}
\\{6.4}t{5.2}{6.8}
\\{12.8}t{5.2}{6.8}
\\{14.4}t{5.2}{6.8}
\\{16.8}t{5.2}{5.6}st{10.8}{5.6}
\\{17.6}t{5.2}{0.8}t{14.8}{1.2}
\\{19.2}t{5.2}{2.4}t{12.0}{3.6}
\\{22.4}t{5.2}{6.8}
\\{22.4}e{9.4}}

\shapepar\bassshape
aarati (ver 1.1) is a python script written by Man\-uel
Gu\-ti\-er\-rez Al\-ga\-ba to produce shape definitions from rough
`ascii art'.  There is no instruction manual, so here are Donald
Arseneau's observations. There is not much of a user interface; look
in proshap.py (which is a plain text file) and see how the various
`test' shapes are defined (note the triple-double quotes).  Choose
one of them, or add a new one, then change the line `test = test3' to select
the desired picture.  Execute `python proshap.py' which will output a
definition of "\bassshape" to the screen and to the file `result.tex'.
The goul\-ish face you see here is the test3 shape.  You should be
aware that the characters in the ascii input are treated as square,
even though they are taller than they are wide, so the output shape
specification will be taller and thinner than the input text.  There
also seems to be a problem with all `bottoms': flat bottoms of text
blocks and of holes are expanded downwards to end at a point.  Compare
this face to the original face in proshap.py.  Warning: These
instructions and observations are probably wrong; the author does not
program in python so can't even read the code properly.  For now, look
for proshap.py bundled with shapepar.sty.
\end{minipage}
\begin{minipage}{0.2\textwidth}

\gdef\bassshape{{9.4}{0}b{9.4}
\\{0}t{5.2}{6.8}
\\{4.0}t{5.2}{6.8}
\\{6.4}t{5.2}{6.8}
\\{12.8}t{5.2}{6.8}
\\{14.4}t{5.2}{6.8}
\\{16.8}t{7.2}{5.6}st{10.8}{8.6}
\\{17.6}t{6.2}{0.8}t{14.8}{8.2}
\\{19.2}t{7.2}{2.4}t{12.0}{7.6}
\\{22.4}t{7.2}{9.8}
\\{22.4}e{9.4}}

\shapepar\bassshape
aarati (ver 1.1) is a python script written by Man\-uel
Gu\-ti\-er\-rez Al\-ga\-ba to produce shape definitions from rough
`ascii art'.  There is no instruction manual, so here are Donald
Arseneau's observations. There is not much of a user interface; look
in proshap.py (which is a plain text file) and see how the various
`test' shapes are defined (note the triple-double quotes).  Choose
one of them, or add a new one, then change the line `test = test3' to select
the desired picture.  Execute `python proshap.py' which will output a
definition of "\bassshape" to the screen and to the file `result.tex'.
The goul\-ish face you see here is the test3 shape.  You should be
aware that the characters in the ascii input are treated as square,
even though they are taller than they are wide, so the output shape
specification will be taller and thinner than the input text.  There
also seems to be a problem with all `bottoms': flat bottoms of text
blocks and of holes are expanded downwards to end at a point.  Compare
this face to the original face in proshap.py.  Warning: These
instructions and observations are probably wrong; the author does not
program in python so can't even read the code properly.  For now, look
for proshap.py bundled with shapepar.sty.
\end{minipage}

\vspace{0pt plus 5cm}
\penalty 0
\vspace{0pt plus -5cm}
\vspace{5cm}
\penalty 9999
\vspace{-5cm}
\subsection{ShapePatch by Christian Gollwitzer}

\def\faceshape{%
{36.8512}%
{-0.519031}b{36.8512}\\%
{-0.259516}t{33.391}{6.92042}\\%
{0.432526}t{30.1038}{13.4948}\\%
{1.55709}t{26.8166}{20.0692}\\%
{3.11419}t{23.7889}{26.1246}\\%
{5.10381}t{21.0208}{31.6609}\\%
{7.43945}t{18.5121}{36.6782}\\%
{10.1211}t{16.3495}{41.0035}\\%
{11.5917}t{15.4412}{11.2024}st{26.6436}{20.5017}st{47.1453}{11.1159}\\%
{11.6782}t{15.3877}{10.6503}t{27.2491}{19.2907}t{47.7509}{10.5638}\\%
{11.7647}t{15.3343}{10.0117}t{27.9412}{17.9066}t{48.4429}{9.9252}\\%
{12.0242}t{15.174}{9.56646}t{28.5467}{16.6955}t{49.0484}{9.47995}\\%
{12.2837}t{15.0137}{9.12121}t{29.1522}{15.4844}t{49.654}{9.0347}\\%
{12.6298}t{14.8}{8.8159}t{29.6713}{14.4464}t{50.173}{8.72939}\\%
{13.0623}t{14.5329}{8.72261}t{30.0317}{13.7255}t{50.5334}{8.6361}\\%
{13.1488}t{14.4968}{8.68656}t{30.1038}{13.5813}t{50.6055}{8.60006}\\%
{13.5813}t{14.3166}{8.43426}t{30.5363}{12.7163}t{51.0381}{8.34775}\\%
{14.1869}t{14.0643}{8.34054}t{30.8824}{12.0242}t{51.3841}{8.25404}\\%
{14.7924}t{13.812}{8.33333}t{31.1419}{11.5052}t{51.6436}{8.24683}\\%
{15.3979}t{13.5597}{8.41263}t{31.3149}{11.1592}t{51.8166}{8.32612}\\%
{16.0035}t{13.3074}{8.57843}t{31.4014}{10.9862}t{51.9031}{8.49193}\\%
{16.1765}t{13.2353}{8.65052}t{31.4014}{10.9862}t{51.9031}{8.56401}\\%
{16.6955}t{13.085}{8.80077}t{31.4014}{10.9862}t{51.9031}{8.71426}\\%
{17.301}t{12.9098}{9.06256}t{31.3149}{11.1592}t{51.8166}{8.97605}\\%
{17.9066}t{12.7345}{9.41085}t{31.1419}{11.5052}t{51.6436}{9.32435}\\%
{18.5121}t{12.5592}{9.84566}t{30.8824}{12.0242}t{51.3841}{9.75915}\\%
{19.1176}t{12.3839}{10.367}t{30.5363}{12.7163}t{51.0381}{10.2805}\\%
{19.4637}t{12.2837}{10.8131}t{30.1903}{13.4083}t{50.692}{10.7266}\\%
{19.5502}t{12.2726}{10.9107}t{30.1038}{13.5813}t{50.6055}{10.8242}\\%
{20.0692}t{12.2061}{11.4098}t{29.6713}{14.4464}t{50.173}{11.3233}\\%
{20.4152}t{12.1617}{11.9732}t{29.1522}{15.4844}t{49.654}{11.8867}\\%
{20.6747}t{12.1285}{12.612}t{28.5467}{16.6955}t{49.0484}{12.5255}\\%
{20.9343}t{12.0952}{13.2508}t{27.9412}{17.9066}t{48.4429}{13.1643}\\%
{21.0208}t{12.0841}{13.954}t{27.2491}{19.2907}t{47.7509}{13.8674}\\%
{21.1073}t{12.073}{14.5706}jt{26.6436}{20.5017}jt{47.1453}{14.4841}\\%
{22.8374}t{11.8512}{50}\\%
{26.2976}t{11.8512}{50}\\%
{29.6713}t{12.2837}{49.1349}\\%
{30.7093}t{12.5842}{41.049}st{53.6332}{7.48497}\\%
{30.7093}t{12.5842}{7.48498}t{53.6332}{7.48497}\\%
{31.0554}t{12.6844}{7.50517}t{53.5001}{7.51789}\\%
{31.0554}t{12.6844}{7.50517}jt{20.1896}{40.8285}\\%
{31.0937}t{12.6955}{7.5074}st{20.2029}{40.804}\\%
{31.1067}t{12.6993}{7.50816}t{20.2189}{33.2615}st{53.4804}{7.52278}\\%
{32.1799}t{13.0099}{7.57079}t{21.5398}{30.5544}t{53.0676}{7.62489}\\%
{32.699}t{13.1602}{7.60107}t{22.3183}{29.1054}t{52.868}{7.67427}\\%
{32.9585}t{13.2353}{7.66652}t{22.7076}{28.3809}t{52.7682}{7.69896}\\%
{33.1315}t{13.3074}{7.68815}t{22.9671}{27.8979}t{52.6782}{7.71684}\\%
{34.083}t{13.7039}{7.80709}t{24.3945}{24.8054}t{52.1834}{7.81517}\\%
{34.7751}t{13.9922}{7.8936}t{25.9007}{22.0881}t{51.8235}{7.88668}\\%
{34.8616}t{14.0283}{7.91397}t{26.0889}{21.7484}t{51.7785}{7.89562}\\%
{35.1211}t{14.1364}{7.97509}t{26.6538}{20.5262}t{51.6436}{7.92243}\\%
{35.5536}t{14.3166}{8.07695}t{27.5952}{18.489}t{51.329}{8.05678}\\%
{36.0727}t{14.5329}{8.19919}t{29.1123}{15.657}t{50.9516}{8.21799}\\%
{36.1592}t{14.5863}{8.20217}t{29.3652}{15.185}t{50.8886}{8.22748}\\%
{36.6782}t{14.9069}{8.2201}t{30.8824}{11.4187}t{50.5112}{8.28437}\\%
{36.7647}t{14.9603}{8.22308}t{31.4446}{10.4815}t{50.4483}{8.29386}\\%
{36.9377}t{15.0672}{8.28058}t{32.5692}{8.60727}t{50.3224}{8.31282}\\%
{37.0242}t{15.1206}{8.30933}t{33.1315}{7.37024}t{50.2595}{8.32231}\\%
{37.1972}t{15.2275}{8.36683}t{34.2561}{4.89619}t{50.0952}{8.37981}\\%
{37.3702}t{15.3343}{8.42433}e{37.8028}t{49.9308}{8.43731}\\%
{38.4948}t{16.0289}{8.79809}t{48.8625}{8.81106}\\%
{38.7543}t{16.1892}{8.94057}t{48.6159}{8.89732}\\%
{39.0138}t{16.3495}{9.08304}t{48.2953}{9.0576}\\%
{40.0519}t{17.1866}{9.45697}t{47.013}{9.50277}\\%
{40.2249}t{17.3262}{9.60168}t{46.7993}{9.57696}\\%
{41.263}t{18.1633}{10.4699}t{45.0198}{10.5194}\\%
{41.436}t{18.3028}{10.7393}t{44.7232}{10.6764}\\%
{41.6955}t{18.5121}{11.1434}t{44.1334}{11.0569}\\%
{42.2145}t{19.0696}{11.8128}t{42.9538}{11.6791}\\%
{42.3875}t{19.2554}{12.1892}t{42.5606}{11.8865}\\%
{42.9066}t{19.8129}{13.3186}t{40.5586}{13.331}\\%
{42.9931}t{19.9058}{13.8096}t{40.2249}{13.5717}\\%
{43.2526}t{20.1845}{15.2826}t{37.8893}{15.6286}\\%
{43.2526}t{20.1845}{15.2826}jt{35.4671}{18.0508}\\%
{44.0311}t{21.0208}{31.6609}\\%
{46.0208}t{23.7889}{26.1246}\\%
{47.5779}t{26.8166}{20.0692}\\%
{48.7024}t{30.1038}{13.4948}\\%
{49.3945}t{33.391}{6.92042}\\%
{49.654}e{36.8512}%
}

\hyphenation{trans-fig shape-patch}

\shapepar\faceshape
ShapePatch is an amazing utility written by Christian Gollwitzer.  It
allows you to simply draw the shape you want with Xfig, and then
convert it to a shapepar shape definition, either by manually running
fig2dev (a component of transfig), or by choosing `Export/Shape' in
Xfig.  Its odd name is indicative of its implementation: it is an
upgrade or `patch' to both the Xfig program and the transfig tool-set.
ShapePatch can be found on \textsc{ctan} under
"graphics/""trans""fig-shape""patch", and includes instructions
(\textsc{readme}) and sample shapes besides the patch itself.  This
smiley face is one of the sample shapes.  In the future, the
ShapePatch utility will be included in transfig, so there will be no
need to install the patch.  Besides enabling a graphical utility
(Xfig) to produce shapepar specifications, the ShapePatch upgrade
opens up great libraries of existing Fig clip-art to be used for
paragraph shapes.  For example, a Canadian flag taken from the Xfig
flag library is one of the examples provided with ShapePatch, for
comparison with the hand-coded one bundled with shapepar.sty. (I
prefer my own version because there is an error in the maple leaf
shape on the Xfig flag library version.  The error is not the fault of
ShapePatch, but ShapePatch faithfully reproduces the faulty shape.)




\section{Shape Syntax}

The syntax rules for <shape_spec> are very specific, and must be
followed closely.  In these rules, "{ }" mean explicit braces, [~]
denote optional parts, <~> surround a keyword that is defined (perhaps
loosely), and $|$ means `or'; do not type [~] <~> or~$|$, but do type "{ }".

\begin{flushleft}
<shape_spec> = "{"<h_center>"}" <lines>  \\[\topsep]
<lines> = <line_spec> [ "\\"<lines> ]
\end{flushleft}

That is, the shape is specified as a single number in braces, followed
by the specifications for the lines, with the lines separated by "\\". The
resulting paragraph will have its <h_center> position centered on the page,
or used as a reference point.  It is a number like 10.5, without explicit
units, but using the same length scale as the
lengths and positions in the <lines>.  Ordinarily, shapepar will
determine the unit length that best fits the text, but will use
a fixed scale when specified for "\shapepar".

The lines in the spec are not lines of text, nor are they the lines
that you would use to draw the shape itself.  They are horizontal
scans across the shape at regular or irregular intervals.  Complex
curved shapes need
many scan lines for accurate rendering, while simple shapes need only
a few.  To determine the line specifications, start by drawing the
shape on paper, then draw a series of horizontal lines across it,
including lines that just touch the top and the bottom of the figure,
and, preferably, lines through each sharp corner.
Each line crosses over pieces of the figure in some region.  These 
intersections of line and figure define a <line_spec>.

\begin{flushleft}
<line_spec> = "{"<v_pos>"}" <segment> [ other <segment>s ]
\end{flushleft}

\noindent
The <v_pos> is the vertical position of the line, increasing from top to
bottom.  Each <line_spec> usually has a position greater than or equal
to that of the previous line, and with all <v_pos>${} > -1000$. The
exception is that between consecutive lines relating to completely
disconnected parts of the figure the <v_pos> may decrease (backspacing).
This allows text to flow from one disconnected area to another in
sequence (see the Canadian flag shape). Each <segment> represents a
region where text will go in the final paragraph; it is the segment of
the horizontal scan line that overlaps the body of the figure.  There
are five types of segment:

\begin{flushleft}
<segment> \ = \ "t{"<pos>"}{"<len>"}" \ $|$ \ "b{"<pos>"}" \ $|$ \ 
          "e{"<pos>"}" \ $|$ \ "s" \ $|$ \ "j"        \\[\topsep]
\begin{tabular}{ll}
  "b{"<pos>"}" & begin text block at a point at horizontal position <pos>\\
  "e{"<pos>"}" & end text at a point at horizontal position <pos>\\
  "t{"<pos>"}{"<len>"}"  & text segment at position <pos> with length <len>\\
  "s"          & split text block (begin a gap)\\
  "j"          & join two text blocks (end a gap)
\end{tabular}
\end{flushleft}

The most common type of segment is "t" (text).  The other types are
degenerate in that they are single points rather than finite segments.
Types "s" and "j" have no explicit position, but they must appear between
text segments, and those texts should abut; e.g.,  "t{3}{2}st{5}{4}"
(text from 3 to 5 and text from 5 to 9).

\bigskip
%\subsection{Diamond shape example}
Let's jump right into a simple example, and the meanings will be
clearer.  
%
\begin{figure}[htbp]
\centering
\begin{picture}(250,160)
\setlength\unitlength{1pt}
%%%%%%%%%%%%%%%%%%%
\thinlines
\put(20,120){\vector(1,0){35} \ $_{\textstyle x}$}
\put(20,120){\vector(0,-1){35}}
\put(27,85){$y$} 
%%%%%%%%%%%%%%%%%%%
\thicklines
\put(120,80){\line(3,4){42}}
\put(120,80){\line(3,-4){42}}
\put(204,80){\line(-3,4){42}}
\put(204,80){\line(-3,-4){42}}
%%%%%%%%%%%%%%%%%%%
\thinlines
\put(162,136){\circle*{3}}
\put(120,80){\circle*{3}}
\put(204,80){\circle*{3}}
\put(162,24){\circle*{3}}
%%%%%%%%%%%%%%%%%%%
\put(162,144){\makebox[0pt]{$(x=3,\ y=0)$}}
\put(117,87){\makebox[0pt][r]{$(0,4)$}}
\put(206,87){\makebox[0pt][l]{$(6,4)$}}
\put(162,10){\makebox[0pt]{$(3,8)$}}
%%%%%%%%%%%%%%%%%%%
\put(105,80){\line(1,0){114}}
\put(105,24){\line(1,0){114}}
\put(105,136){\line(1,0){114}}
%%%%%%%%%%%%%%%%%%%
\end{picture}
%
%                   (x=3,y=0)
%                       .
%
%  +---> x
%  !          (0,4) .       . (6,4)
%  !
%  V y
%                       .
%                     (3,8)
%
\vspace{-10pt}
\caption{Diamond shape, showing vertex locations and scan lines.}
\label{fig:diamond}
\end{figure}
%
A rhombus `diamond' shape can have the four vertices, with
coordinates shown in figure~\ref{fig:diamond}.
This shape can be exactly specified by just three scan lines passing
through the vertices.  The intersections of the scan lines with the
shape's edges occurs at the vertices and so the shape specification is:
\begin{flushleft}
\begin{tabular}{@{}ll}
"{3}"          &  <h_center>: $x = 3$\\
"{0}b{3}\\"    &  text block begins at point $y=0, x=3$\\
"{4}t{0}{6}\\" &  scan (at $y=4$) crosses text (len~6) starting at $x=0$\\
"{8}e{3}"      &  text block ends at point $y=8, x=3$
\end{tabular}
\end{flushleft}
Other specification lines, such as
\begin{flushleft}
"{6}t{1.5}{3}\\"
\end{flushleft}
could be inserted, but would make no difference~-- the shape is
interpolated linearly between scan lines.

\medskip
%\subsection{Valid specs}

Every block of text must start with a "b" specifier and end with an "e"
spec on some line below.  Every segment specified by "t" must have a
length greater than zero. If two blocks of text merge to form one (like
at the notch of a heart shape) there should be a "j" spec at the point of
junction.  If one block bifurcates or splits (like at the top of a hole in
a doughnut) there should be an "s" spec.

Thus, the first line for any valid shape description must consist
of only "b" segment descriptors; the last line can only have "e" type
descriptors.  Although the definition of the units is arbitrary, the
numbers should range in magnitude from ${\sim}\,0.1$ to 100 to avoid
numeric overflows and underflows.

If there are errors in the format of the specification, "\shapepar"
might complain with the error message
\begin{flushleft}
Shaped Paragraph Error:  Error in specification.  Check carefully!
\end{flushleft}
At this point you may as well type "x" or "e", to exit from \TeX,
as there is very little chance that {\TeX} will continue successfully.
You might also get one of \TeX's regular error messages, like
\begin{flushleft}
Illegal unit of measure (pt inserted).\\[3pt]
or\\[3pt]
Missing number, treated as zero.
\end{flushleft}
or you might get no error message at all, just ridiculous formatting.
Check shape syntax carefully against the rules and the examples before
running them through \TeX.

\medskip
%\subsection{Square example}

What to do if the figure does not start at a point~-- if it has a flat
top?  It can start at a single point, but have the next scan line at
the same vertical position!  A square paragraph is specified by:
\begin{flushleft}
\begin{tabular}{@{}ll}
  "{1}"          &  centerline is at $x=1$\\
  "{0}b{0}\\"    &  begin at $(0,0)$\\
  "{0}t{0}{2}\\" &  text at $y=0$, width${}=2$\\
  "{2}t{0}{2}\\" &  text at $y=2$, width${}=2$\\
  "{2}e{1}"      &  end at $(1,2)$
\end{tabular}
\end{flushleft}


%\subsection{Heart example}

Now let's get more ambitious.  A heart shape must have two simultaneous
beginnings, a short stretch where there are two separated text areas
ending with a join,
whereafter there is just one block of text leading to the final
bottom point.  Figure \ref{fig:heart} shows the heart-shape
specification.
\begin{figure}[tbp]
\begin{small}
\begin{minipage}[b]{21em}
\renewcommand{\baselinestretch}{.9}\selectfont
\begin{verbatim}
\newcommand\heartshape{
{20}{0} b{13.32}         b{26.68}
\\{.14} t{10.12}{4.42}   t{25.46}{4.42}
\\{.7}  t{9.14}{7.16}    t{23.7}{7.16}
\\{1.4} t{8.4}{9.02}     t{22.58}{9.02}
\\{2.1} t{7.82}{10.42}   t{21.76}{10.42}
\\{2.8} t{7.36}{11.58}   t{21.06}{11.58}
\\{3.5} t{6.98}{12.56}   t{20.46}{12.56}
\\{4.2} t{6.68}{13.32} j t{20}{13.32}
\\{4.9}   t{6.48}{27.04}
\\{5.6}   t{6.34}{27.32}
\\{6.3}   t{6.28}{27.44}
\\{7}     t{6.26}{27.48}
\\{7.7}   t{6.27}{27.46}
\\{8.4}   t{6.32}{27.36}
\\{9.1}   t{6.4}{27.2}
\\{9.8}   t{6.52}{26.96}
\\{10.5}  t{6.68}{26.64}
\\{11.9}  t{7.12}{25.76}
\\{13.3}  t{7.72}{24.56}
\\{14.7}  t{8.51}{22.98}
\\{16.1}  t{9.5}{21}
\\{17.5}  t{10.69}{18.62}
\\{18.9}  t{12.08}{15.84}
\\{20.3}  t{13.7}{12.6}
\\{21.7}  t{15.62}{8.76}
\\{22.4}  t{16.7}{6.6}
\\{23.1}  t{17.87}{4.26}
\\{24.6}  e{20}
}
\end{verbatim}
\end{minipage}%
\hspace*{0pt plus 1fill minus \linewidth}%
\addtolength{\linewidth}{-13em}
\begin{minipage}[b]{\linewidth}
\footnotesize
\def~{\,}
\heartpar{%
\mbox{ }In faith, I do not love thee with mine eyes,
~For they in thee a thousand errors note,
~But is my heart that loves what they despise,
~Who in despite of view is pleased to dote;
~Nor are mine ears with thy tongue's tune delighted,
~Nor tender feeling to base touches prone,
~Nor taste nor smell, desire to be invited
~To any sensual feast with thee alone;
~But my five wits nor my five senses can
~Dissuade one foolish heart from serving thee,
~Who leaves unswayed the likeness of a man,
~Thy proud heart's slave and vassal wretch to be;
~Only my plague thus far I count my gain,
~That she that makes me sin awards me pain.}
~~~~
\end{minipage}
\end{small}
\caption{Specification for the heart shape, and an example.}
\label{fig:heart}
\end{figure}
Find the two "b" specifiers at the beginning, and find the "j" a few
lines below; notice that above the "j" there are two segments per line,
but only one below it~-- the two lobes join at the "j"
point: 20.  I drew this heart freehand, and  measured
lengths from the sketch, so you should be able to do better! The spec
has many scan lines so that the smooth curves are preserved, but there
are probably more lines than necessary.

\medskip
%\subsection{Holes -- nut shape}

Text shapes can have holes.  For example, a doughnut-shape would have a "b"
on the first line, followed by some lines with a single "t",  then a line with
"t s t" at the start of the hole. The hole is represented by lines with two
"t" specs~-- the gap between them is the hole.  A line with  "t j t" ends the
hole.  There are more lines with single "t", and then an "e" line to end
with.  Such a doughnut is used by the "\CDlabel" shape, but the
example given in Figure~\ref{fig:nut} is a nut. Not a doughnut, but a hex-nut (for a
machine screw or bolt)~-- a regular hexagon with a circular hole in the center.
The hexagon is flat on top and bottom so the specification begins and
ends like the square shape. The circle is rendered as a 24-gon, beginning
with a split ("s") of the surrounding text and ending with a join ("j").
If the spacing of the scan lines looks odd, it is because the hexagon alone
would need just 5 scan lines (at only 3 distinct locations), but the circle
needs many; the points on the circle are at 15-degree intervals.
\begin{figure}[htbp]
\renewcommand{\baselinestretch}{.9}
\begin{verbatim}
\newcommand\nutshape{
{0}
{0}     b{0}\\
{0}     t{-12.5}{25}\\
{11.65} t{-19.23}{19.23} s t{0}{19.23}\\
{11.99} t{-19.42}{16.835}  t{2.59}{16.835}\\
{12.99} t{-20}{15}         t{5}{15}\\
{14.58} t{-20.92}{13.85}   t{7.07}{13.85}\\
{16.65} t{-22.11}{13.45}   t{8.66}{13.45}\\
{19.06} t{-23.51}{13.85}   t{9.66}{13.85}\\
{21.65} t{-25}{15}         t{10}{15}\\
{24.24} t{-23.51}{13.85}   t{9.66}{13.85}\\
{26.65} t{-22.11}{13.45}   t{8.66}{13.45}\\
{28.72} t{-20.92}{13.85}   t{7.07}{13.85}\\
{30.31} t{-20}{15}         t{5}{15}\\
{31.31} t{-19.42}{16.835}  t{2.59}{16.835}\\
{31.65} t{-19.23}{19.23} j t{0}{19.23}\\
{43.3}  t{-12.5}{25}\\
{43.3}  e{0}
}
\end{verbatim}
\vspace{-\topsep}
\caption{Specification for the nut shape.
(The definition in shapepar.sty is the same except that
the spaces are removed.)}
\label{fig:nut}
\end{figure}

\section{Configuration}

There are several parameters that control details of how "\shapepar"
functions, varying from those (one) that will be set often, to some that
are cryptic, and require editing shapepar.sty itself.  Here we will
explain some of those operational details and the parameters that
control them.

\subsection{Cut-out separation}

The cutout separation is applied by looking at the paragraph shape
expanded by "\cutoutsep" in `all' directions, to give the same gap at
every slope.  That is a white lie.  The shape's `"\parshape"' is
actually regenerated at a finer line-spacing (but the same scale factor
as for the shaped paragraph), and the leftmost (or rightmost) position
of each finer-parshape line is used to exclude an octagon (as an
approximation to a circle) centered on that position. The octagon is not
even regular, but is extended vertically by
$\hbox{"\cutoutsepstretch"}\times\hbox{"\baselineskip"}$ to allow
for the height and depth of characters.  Figure~\ref{fig:cutoutsep}
illustrates these parameters.

\begin{figure}[htbp]
\small
\setlength\cutoutsep{53pt}
\renewcommand\cutoutsepstretch{1.0}
\cutout{r}(-\cutoutsep,7\baselineskip)
\shapepar[1pt]\squareshape
\setlength\unitlength{1pt}\thinlines
\begin{picture}(0,0)
\put(0,0){\makebox[1pt]{A}}
%
 \xdef\CUa{\Pointless\cutoutsep}
 \lineskiplimit=.414\cutoutsep
 \xdef\CUb{\Pointless\lineskiplimit}
 \advance\lineskiplimit\cutoutsepstretch\baselineskip
 \xdef\CUc{\Pointless\lineskiplimit}
 \advance\lineskiplimit .586\cutoutsep
 \xdef\CUd{\Pointless\lineskiplimit}
 \lineskiplimit=.586\cutoutsep
 \xdef\CUe{\Pointless\lineskiplimit}
 \lineskiplimit=\cutoutsep
 \advance\lineskiplimit -5pt
 \xdef\CUf{\Pointless\lineskiplimit}
 \advance\lineskiplimit\cutoutsepstretch\baselineskip
 \xdef\CUg{\Pointless\lineskiplimit}
%
% Outer Octagon
\put(-\CUa,-\CUc){\line(0,1){\CUc}}
\put(-\CUa,0){\line(0,1){\CUc}}
\put(\CUa,-\CUc){\line(0,1){\CUc}}
\put(\CUa,0){\line(0,1){\CUc}}
\put(-\CUa,\CUc){\line(1,1){\CUe}}
\put(-\CUa,-\CUc){\line(1,-1){\CUe}}
\put(\CUa,\CUc){\line(-1,1){\CUe}}
\put(\CUa,-\CUc){\line(-1,-1){\CUe}}
\put(-\CUb,\CUd){\line(1,0){\CUb}}
\put(0,\CUd){\line(1,0){\CUb}}
\put(-\CUb,-\CUd){\line(1,0){\CUb}}
\put(0,-\CUd){\line(1,0){\CUb}}
%
% Inner Octagon
\put(-\CUa,\CUb){\line(1,1){\CUe}}
\put(-\CUa,-\CUb){\line(1,-1){\CUe}}
\put(\CUa,\CUb){\line(-1,1){\CUe}}
\put(\CUa,-\CUb){\line(-1,-1){\CUe}}
\put(-\CUb,\CUa){\line(1,0){\CUb}}
\put(0,\CUa){\line(1,0){\CUb}}
\put(-\CUb,-\CUa){\line(1,0){\CUb}}
\put(0,-\CUa){\line(1,0){\CUb}}
%
% Distance arrows
\put(-10,0){\vector(0,1){\CUd}}
\put(-10,0){\vector(0,-1){0}}
\put(10,0){\vector(0,1){\CUa}}
\put(10,0){\vector(0,-1){0}}
\put(0,0){\vector(-1,0){\CUa}}
\put(0,0){\vector(1,0){0}}
\put(0,-\CUf){\vector(-1,0){\CUb}}
\put(0,-\CUf){\vector(1,0){\CUb}}
\put(\CUf,0){\vector(0,1){\CUc}}
\put(\CUf,0){\vector(0,-1){\CUc}}
%
% Distance labels
\put(\CUf,0){\makebox[0pt]{$c$ \ \ }}
\put(0,-\CUf){\makebox[0pt]{\shortstack{$b$\\~}}}
{
 \unitlength=.5pt % half-distances for centered labels
 \put(-\CUa,-15){\makebox[0pt]{$a$}}
 \put(-12,\CUd){$d$}
 \put(28,\CUa){$a$}
}
\end{picture}
\par
\refstepcounter{figure}\label{fig:cutoutsep}%
Figure \thefigure: This is an example with a tiny (unrecognizable)
square shaped `paragraph' and a large cutout separation.  Here "\cutoutsep" is
set to \the\cutoutsep, and "\cutoutsepstretch" is \cutoutsepstretch. Since the
`paragraph' is so small (`A'), and the separation so great, the cutout
is dominated by the octagonal expansion of the shape which
is drawn for "\cutoutsep" alone and for the full exclusion zone
including "\cutoutsepstretch".  The labelled distances are
as follows:
$a={}$"\cutoutsep"; $b=0.828$\,"\cutoutsep";
$c=b+2\times\mbox{"\cutoutsepstretch"}\times\mbox{"\baselineskip"}$;
$d=a+\mbox{"\cutoutsepstretch"}\times\mbox{"\baselineskip"}$.
The exclusion zone (outer octagon) applies to the baseline position
of each paragraph line, so the characters do intrude into the octagon
from above and below.  The vertical expansion factor "\cutoutsepstretch"
should be chosen to just counteract this effect, leaving a symmetric
border of white space (inner octagon).
In ordinary use, with a larger shaped paragraph and a smaller
separation, the expansion of the shape should appear equal in all directions
(circular); it is only in this contrived example that it is revealed
as an octagon. 
\par
\end{figure}

A good place to set these parameters is between "\cutout"
and "\shapepar" so they apply locally to just that instance.

\configpar{cutoutsep}
 {Distance separating shaped paragraph and the surrounding cut-out text.}
 {length (dimen register)}
 {12\,pt}
 {"\setlength"}
 {"\setlength{\cutoutsep}{1cm}" \ or \ "\cutoutsep= 5pt"}

\configpar{cutoutsepstretch}
 {Vertical extension of cutout gap, given as a fraction of "\baselineskip",
  which accounts for the height and depth of characters.  Note that,
  even with a good setting for "\cutoutsepstretch", the separation
  above and below the shaped paragraph may not match the side spacing
  simply because the cutout text has rigid baseline skips.
  \interlinepenalty 1000\nopagebreak[0]}
 {macro (`command', but really `data')}
 {.5}
 {"\renewcommand" or "\def"}
 {"\renewcommand{\cutoutsepstretch}{.75}"}

\configpar{RefineBaselines}
 {Fineness of cut-out matching to lines (number of reference points
  per line of cut-out text).}
 {integer constant (not a \LaTeX\ counter)}
 {3}
 {"\renewcommand", "\def", "\chardef", "\mathchardef" etc.}
 {\rightskip\fill"\renewcommand{\RefineBaselines}{2}" \ or \ 
        \hbox{"\chardef\RefineBaselines= 4"}}


\subsection{Scale length optimization}

When "\shapepar" is used without an explicit scale length, it must
determine a scale that allows the given text to fill the shape.
It makes a first guess by comparing the length of the text with
the area of the shape specification, and then typesets the paragraph
at that scale. If the text does not fit well, then "\shapepar"
changes the scale and tries again.  It will make as many as
"\ScaleMaxTries" trial paragraphs:

\configpar{ScaleMaxTries}
 {How many times will "\shapepar" try to get the size of the paragraph?}
 {integer constant (not a \LaTeX\ counter)}
 {9}
 {"\renewcommand", "\def", "\chardef", "\mathchardef" etc.}
 {"\renewcommand{\ScaleMaxTries}{7}" \ or \ 
    \hbox{"\chardef\ScaleMaxTries= 5"}}

\medskip
It is quite common that text will not esaily fit a shape even when
the best scaling is chosen, so it is counter-productive to set a very
high value for "\ScaleMaxTries".

There are many other numbers explicitly coded into shapepar.sty that
control the scale-optimization process: choosing the initial guess,
choosing the initial step size (`dscale'), changing dscale by different
factors (`fac'), fitting the text to the paragraph shape, and more.
These are accompanied by the comment `optimize', and they can be
changed by editing shapepar.sty.

\subsection{Applying the shape}

"\shapepar" cheats a bit when the horizontal gap between two bits of text
is small (like down in the notch of "\heartpar").  When the gap is less
than an interword space it is eliminated, and the texts are joined; when
it is somewhat larger it is expanded to give it more visibility.  Likewise,
when a segment of text is too small, it is eliminated.  There are three
parameters to control this behavior. Although they are lengths, they are
all macros (commands) rather than dimen registers in order to use
font-size-based length units (em).

\configpar{SmallestGap}
 {Smallest gap size allowed; smaller gaps will be eliminated (text joined).}
 {macro (command) giving a length}
 {.4\,em}
 {"\renewcommand" or "\def"}
 {"\renewcommand{\SmallestGap}{.5em}"}

\configpar{SmallGap}
 {Small gap size; smaller gaps will be enlarged.}
 {macro (command) giving a length}
 {1\,em}
 {"\renewcommand" or "\def"}
 {"\renewcommand{\SmallGap}{2em}"}

\configpar{SmallestSegment}
 {Smallest segment allowed; smaller will be omitted.}
 {macro (command) giving a length}
 {.2\,em}
 {"\renewcommand" or "\def"}
 {"\renewcommand{\SmallestSegment}{10pt}"}

\subsection{Feedback}

Since the processing is slow, some messages will be displayed to show how
things are going.  \LaTeX\ users can disable this feedback by loading
shapepar with
"\usepackage""[quiet]""{shapepar}", or they can get more verbose
messages by requesting "\usepackage""[noisy]""{shapepar}".  There are
even more verbose messages that can be activated by removing the \%
that hides them, but they are only useful for debugging shapepar.sty
itself.

% Kludge strike-through because I don't want to load more packages

\section[Future improvements]%
 {\begin{tabular}[t]{l}Deficiencies\\[-1.5ex]\hline\end{tabular}
  Future Improvements}

\begin{itemize}
\item The shape is followed by the text baselines, and this makes the
shaped paragraph marginally taller; mis-fitting of baselines makes the
paragraph squatter.

\item Exact placement of CD-labels on label stock and of graphics in a
cut-out requires a box reference-point defined by the shape alone, 
not the text.  But portions of the text spill out of the shape and 
might go off the label.  How to handle best?

\item The restriction to a single paragraph is a pain, especially
for fixed-scale shapes.  The pre-boxing and un-boxing of the text 
forces a single paragraph without displays.  Omitting pre-boxing is
easiest with fixed-scale, but even then there may be multiple
typesettings.  Global assignments will accumulate and my register
re-use will cause conflicts.

\item The restrictions in cut-out text are likewise a problem.
Should upgrade wrapfig.sty and merge some code.

\item Discrete segments and fixed-scale suggest an application to 
magazine or poster layout.  For this to be useful, excess text 
should be able to overflow onto the next page, and complex text 
(multiple paragraphs, lists, and displays) must be accepted.

\end{itemize}

\end{document}
