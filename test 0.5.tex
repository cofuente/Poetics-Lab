\documentclass[draft]{article}
\usepackage{shapepar}

\title{Ecopoetics Groundwork}
\author{Ira Livingston}
\date{v0.5 \ Mar 2022}

\begin{document}


\maketitle
\subsection{example heart with shapepar}

\shapepar\heartshape
Ecopoetics Groundwork is a conceptual primer for the transdisciplinary enterprise of ecopoetics.  Primarily relying on examples rather than abstractions as such, the book aims to do the brain-rewiring required to ground ecopoetics in an anarchist philosophy of open systems.  It explores reparative practices that work on and through language and its kinship with complex and evolving ecologies, drawing on the Kabbalistic practice of tikkun (repair), which links activism with more arcane work of meaning-making.  Starting from Wittgenstein's mandate that "the whole of language has to be thoroughly plowed up," groundwork refers to an overturning and opening up of conceptual and imagistic grammar that deconstructs modernist dualities among nature, humanity, and divinity.  After this introduction, the chapters follow ecopoetics through the realms of language (two), visual art and architecture (three), science (four), religion and mysticism (five), and poetry (six), accompanied by key poetic texts at every turn.

\end{document}