\documentclass[draft]{article}
\usepackage{shapepar}
\usepackage{microtype}

\title{Ecopoetics Groundwork}
\author{Ira Livingston}
\date{v0.8 \ Mar 2022}

\begin{document}
\maketitle
\subsection{using xfig figures}
% https://tex.stackexchange.com/questions/32997/very-specific-paragraph-shape
\def\coatpar#1{\shapepar{\coatshape}#1\par}
\def\coatshape{%
{25.1761}%
{0.176056}b{0.176056}\\%
{0.176056}t{0.176056}{50}\\%
{1.05634}t{0.176056}{0.880282}st{1.05634}{49.1197}\\%
{2.28873}t{0.176056}{0.352113}t{2.46479}{47.7113}\\%
{3.52113}t{0.176056}{1.46262}t{3.87324}{46.3028}\\%
{23.5915}t{0.176056}{19.5481}t{26.8109}{23.3652}\\%
{25.1761}t{0.459004}{20.6929}t{28.6217}{21.5543}\\%
{26.0563}t{0.616197}{21.3289}t{29.6278}{7.16801}st{36.7958}{13.1756}\\%
{28.5211}t{1.05634}{23.1098}t{32.4447}{1.88632}t{39.2606}{10.1376}\\%
{29.4014}t{1.33929}{23.6201}e{33.4507}t{40.1408}{9.05257}\\%
{32.7465}t{2.41449}{25.5591}t{36.6503}{11.7652}\\%
{33.4507}t{2.64085}{25.9673}t{35.9155}{12.1144}\\%
{34.331}t{3.14767}{26.2537}t{36.9312}{10.6166}\\%
{36.7958}t{4.56679}{22.3698}b{32.7465}t{39.7752}{6.42283}\\%
{37.5}t{4.97226}{21.2601}t{32.0423}{1.39245}t{40.5878}{5.22462}\\%
{39.2606}t{5.98592}{22.3592}t{30.2817}{4.87356}t{42.6192}{2.22908}\\%
{40.1408}t{6.62612}{22.7753}jt{29.4014}{6.61412}t{43.6349}{0.73131}\\%
{40.3169}t{6.75416}{29.4334}t{43.838}{0.377263}\\%
{44.5423}t{9.82715}{30.4898}jt{40.3169}{0.27666}\\%
{45.0704}t{10.2113}{29.9296}\\%
{49.2958}t{15.1408}{20.1344}\\%
{50.8803}t{17.5593}{15.8914}\\%
{52.6408}t{20.2465}{9.31246}\\%
{54.2254}e{26.0563}%
}

\coatpar{I begin with a bit of a poem, the way a rabbi or preacher would start with a biblical verse. Today's sermon is drawn from Wallace Stevens' massively canonical poem "The Idea of Order at Key West," in which two men philosophize as they watch and listen to a woman singing at the seaside, and they are struck by the sense that. . . there was no world for her Except the one she sang and, singing, made. The men are trying to come to terms with what, if anything, the singing (standing for art, language and consciousness generally) does to the world-- its relationship with the sea, and the question of who dances to whose tune. As in Stevens' poem "Anecdote of the Jar," in which the simple placement of an empty glass jar on a hill has somehow organized the wilderness around it and has taken "dominion everywhere," the effect of her singing is both vanishingly subtle and total; the sea and the night sky are harmonized, enchanted, and thrown into mystical perspective by it. What can be the effect of human meaning-making on the world, and how do language and art participate in shaping it? An open, high-stakes question for us in the 21st century, and for ecopoetics. Part of the point of Stevens' poem seems to be how philosophy-- which Stevens codes as masculine-- falls short of art-- which he codes as feminine-- but the poet, by folding philosophy back into art in the form of the poem, manages to perform a transcending synthesis. It's an old Wordsworthian move: a kind of dialectical masculinism, starting with the binary distinctions of culture and nature, singer and sea. The poet is the woman singing and the men philosophizing. }
\end{document}