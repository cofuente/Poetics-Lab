\documentclass[draft]{article}
\usepackage{shapepar}

\title{Ecopoetics Groundwork}
\author{Ira Livingston}
\date{v0.6 \ Mar 2022}

\begin{document}


\maketitle
\subsection{proshap.py by Man\-uel Gu\-ti\-er\-rez Al\-ga\-ba}
\gdef\bassshape{{9.4}{0}b{9.4}
\\{0}t{7.6}{6.8}
\\{0.8}t{5.2}{11.6}
\\{1.6}t{4.4}{13.2}
\\{2.4}t{4.0}{14.0}
\\{3.2}t{3.6}{4.933333}st{8.533333}{4.933333}st{13.46666}{4.933333}
\\{4.0}t{3.6}{2.4}t{9.6}{2.8}t{16.4}{2.4}
\\{6.4}t{3.6}{2.0}t{10.0}{2.0}t{16.8}{2.0}
\\{8.8}t{3.6}{2.4}t{9.6}{1.4}st{11.0}{1.4}t{16.4}{2.0}
\\{11.2}t{3.6}{3.4}jt{7.0}{3.4}t{11.6}{3.4}jt{15.0}{3.4}
\\{12.8}t{4.0}{6.0}t{12.0}{6.0}
\\{14.4}t{4.4}{5.2}t{12.4}{5.2}
\\{16.0}t{4.8}{6.0}jt{10.8}{6.0}
\\{16.8}t{5.2}{5.6}st{10.8}{5.6}
\\{17.6}t{5.6}{0.8}t{14.8}{1.2}
\\{19.2}t{6.0}{2.4}t{12.0}{3.6}
\\{20.8}t{6.4}{4.4}jt{10.8}{4.4}
\\{22.4}e{9.4}}


\shapepar\bassshape
proshap.py (ver 1.1) is a python script written by Man\-uel
Gu\-ti\-er\-rez Al\-ga\-ba to produce shape definitions from rough
`ascii art'.  There is no instruction manual, so here are Donald
Arseneau's observations. There is not much of a user interface; look
in proshap.py (which is a plain text file) and see how the various
`test' shapes are defined (note the triple-double quotes).  Choose
one of them, or add a new one, then change the line `test = test3' to select
the desired picture.  Execute `python proshap.py' which will output a
definition of "\bassshape" to the screen and to the file `result.tex'.
The goul\-ish face you see here is the test3 shape.  You should be
aware that the characters in the ascii input are treated as square,
even though they are taller than they are wide, so the output shape
specification will be taller and thinner than the input text.  There
also seems to be a problem with all `bottoms': flat bottoms of text
blocks and of holes are expanded downwards to end at a point.  Compare
this face to the original face in proshap.py.  Warning: These
instructions and observations are probably wrong; the author does not
program in python so can't even read the code properly.  For now, look
for proshap.py bundled with shapepar.sty.

\subsection{proshap.py by Man\-uel Gu\-ti\-er\-rez Al\-ga\-ba}
\begin{minipage}{0.2\textwidth}
\gdef\bassshape{{9.4}{0}b{9.4}
\\{0}t{5.2}{6.8}
\\{4.0}t{5.2}{6.8}
\\{6.4}t{5.2}{6.8}
\\{12.8}t{5.2}{6.8}
\\{14.4}t{5.2}{6.8}
\\{16.8}t{5.2}{5.6}st{10.8}{5.6}
\\{17.6}t{5.2}{0.8}t{14.8}{1.2}
\\{19.2}t{5.2}{2.4}t{12.0}{3.6}
\\{22.4}t{5.2}{6.8}
\\{22.4}e{9.4}}

\shapepar\bassshape
This (ver 1.1) is a python script written by Man\-uel
Gu\-ti\-er\-rez Al\-ga\-ba to produce shape definitions from rough
`ascii art'.  There is no instruction manual, so here are Donald
Arseneau's observations. There is not much of a user interface; look
in proshap.py (which is a plain text file) and see how the various
`test' shapes are defined (note the triple-double quotes).  Choose
one of them, or add a new one, then change the line `test = test3' to select
the desired picture.  Execute `python proshap.py' which will output a
definition of "\bassshape" to the screen and to the file `result.tex'.
The goul\-ish face you see here is the test3 shape.  You should be
aware that the characters in the ascii input are treated as square,
even though they are taller than they are wide, so the output shape
specification will be taller and thinner than the input text.  There
also seems to be a problem with all `bottoms': flat bottoms of text
blocks and of holes are expanded downwards to end at a point.  Compare
this face to the original face in proshap.py.  Warning: These
instructions and observations are probably wrong; the author does not
program in python so can't even read the code properly.  For now, look
for proshap.py bundled with shapepar.sty.
\end{minipage}
\begin{minipage}{0.2\textwidth}

\gdef\bassshape{{9.4}{0}b{9.4}
\\{0}t{5.2}{6.8}
\\{4.0}t{5.2}{6.8}
\\{6.4}t{5.2}{6.8}
\\{12.8}t{5.2}{6.8}
\\{14.4}t{5.2}{6.8}
\\{16.8}t{7.2}{5.6}st{10.8}{8.6}
\\{17.6}t{6.2}{0.8}t{14.8}{8.2}
\\{19.2}t{7.2}{2.4}t{12.0}{7.6}
\\{22.4}t{7.2}{9.8}
\\{22.4}e{9.4}}

\shapepar\bassshape
aarati (ver 1.1) is a python script written by Man\-uel
Gu\-ti\-er\-rez Al\-ga\-ba to produce shape definitions from rough
`ascii art'.  There is no instruction manual, so here are Donald
Arseneau's observations. There is not much of a user interface; look
in proshap.py (which is a plain text file) and see how the various
`test' shapes are defined (note the triple-double quotes).  Choose
one of them, or add a new one, then change the line `test = test3' to select
the desired picture.  Execute `python proshap.py' which will output a
definition of "\bassshape" to the screen and to the file `result.tex'.
The goul\-ish face you see here is the test3 shape.  You should be
aware that the characters in the ascii input are treated as square,
even though they are taller than they are wide, so the output shape
specification will be taller and thinner than the input text.  There
also seems to be a problem with all `bottoms': flat bottoms of text
blocks and of holes are expanded downwards to end at a point.  Compare
this face to the original face in proshap.py.  Warning: These
instructions and observations are probably wrong; the author does not
program in python so can't even read the code properly.  For now, look
for proshap.py bundled with shapepar.sty.
\end{minipage}


\end{document}